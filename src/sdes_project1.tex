\documentclass[12pt, a4paper]{report}

\usepackage{graphicx}
\usepackage{authblk}
\usepackage{hyperref}
\usepackage{float}

\title{Pendulum with Friction}
\author{Asim Ukaye}
\affil{Roll. No.: 130010012}
\date{\today}

\begin{document}
\maketitle

\begin{abstract}
This report elaborates the simulation of the motion of a pendulum using python programming. The pendulum taken into consideration over here is a simple pendulum with a point mass attached to a mass-less string. The report is divided into 3 sections. Section 1 deals with the physics of the problems and the equations involved behind it. Section 2 shows the plots that were obtained and the problem conditions that were chosen. Section 3 highlights the references and the requirements needed to reproduce this project.  %contain $\dot{\theta}$ and $\theta$ plots with respect to time. 
%The python code generated these plots by taking the mass, length and damping constant of the pendulum and $\dot{\theta}$ and $\theta$ as arguments for the function and these can be changed which will automatically reflect in the pdf report once the make file in run.%

\end{abstract}


\section*{Physics behind the problem}
\subsection*{Equation of Motion}
Equation of motion for a mass-less pendulum with a bob at the end is 

\begin{equation}
\label{equation1}
\ddot{\theta} + \frac{c}{m}\dot{\theta} + \frac{g}{L}sin\theta = 0
\end{equation}

where,\\
c = damping constant in $\frac{kg}{s}$\\
m = mass of the bob in $kg$\\
g = acceleration due to gravity which is taken to be $9.8\frac{m}{s^2}$\\
L = length of the pendulum in $m$\\
$\theta$ = Angular displacement of the pendulum from the vertical in $rad$\\
$\dot{\theta}$ = Angular velocity of the pendulum in $\frac{rad}{s}$\\
$\ddot{\theta}$ = Angular acceleration of the pendulum in $\frac{rad}{s^2}$\\

In equation \ref{equation1} using small angle approximation the equation can be re-written as

\begin{equation}
\label{equation2}
\ddot{\theta} + \frac{b}{m}\dot{\theta} + \frac{g}{L}\theta = 0
\end{equation}

\subsection*{Solutions}
The equation can be solved assuming the general solution to be $\theta = e^{\lambda}$
Substituting this in equation \ref{equation2} we get

\begin{eqnarray}
\lambda^2 + \frac{b}{m}\lambda + \frac{g}{L} = 0\\
\lambda_1 = \frac{-\frac{b}{m} + \sqrt{(\frac{b}{m})^2 - \frac{4g}{L}}}{2}\\
\lambda_2 = \frac{-\frac{b}{m} - \sqrt{(\frac{b}{m})^2 - \frac{4g}{L}}}{2}\\
\end{eqnarray}

Solution can be written as 
\begin{equation}
\label{equation3}
\theta = a_1e^{\lambda_1} + a_2e^{\lambda_2}
\end{equation}

Using initial conditions for $\theta$ and $\dot{\theta}$, constants $c_1$ and $c_2$ in equation \ref{equation3} can be written as
\begin{eqnarray}
a_1 = \frac{\dot{\theta}(0) - \lambda_2\theta(0)}{\lambda_1 - \lambda_2}\\
a_2 = \frac{\lambda_1\theta(0) - \dot{\theta}(0)}{\lambda_1 - \lambda_2}
\end{eqnarray}

This solution will work for both underdamped and overdamped cases. It won't work for critical damped case because $\lambda_1=\lambda_2$ in critical damped case. However for practical values of damping constant `c' the pendulum will always be in under-damped condition so the critical damped case is ignored where the above solution won't work.

\section*{}{Results and Plots}
The result from the code is obtained by using the following parameters:\\\\
mass = 1 $kg$\\
c = 0.4 $\frac{kg}{s}$\\
L = 1 $m$\\
$\theta$ = 1 $rad$\\
$\dot{\theta}$ = 0 $\frac{rad}{s}$\\

Obtained graphs are as under:

\begin{figure}[H]
\label{position}
\includegraphics[width = \textwidth]{Angular_Position_vs_time.png}
\caption{$\theta$ vs time}
\end{figure}

\begin{figure}[H]
\label{velocity}
\includegraphics[width = \textwidth]{Angular_Velocity_vs_time.png}
\caption{$\dot{\theta}$ vs time}
\end{figure}

\section*{References and Requirements}
\begin{itemize}
\item Public git repository with open source code can be found at \url{https://github.com/asimukaye/sdes_project1.git}
\item Physics reference at \url{http://nrich.maths.org/content/id/6478/Paul-not%20so%20simple%20pendulum%202.pdf}
\item Ipython version 5.1.0 notebook or higher is required to run the Ipython code
\item Python 2.7.6 version is required to run the Python code
\item ffmpeg needs to be installed to save the animation

\end{itemize}

\end{document}